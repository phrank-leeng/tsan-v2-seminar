%%%%%%%%%%%%%%%%%%%%%%%%%%%%%%%%%%%%%%%%%%%%%%%%%%%%%%%%%%%%%%%%%%%%%%%%%%%%%%%%%%%%%%%%%%%%%%%%%%%%%%%%%%%%%%%%%%%%%%%%%%%%%%%%%%%%%%%
%%| !!! TEMPLATE INSPIRED BY https://github.com/ocjojo/hska-latex-template                  !!! |%%
%%| !!! HsKa_old.jpg CAN BE FOUND HERE: https://de.wikipedia.org/wiki/Datei:Hska_logo.svg   !!! |%%
%%| !!! HsKa_new.png WAS CREATED BY USAGE OF IMAGERY FROM: https://www.h-ka.de/intern       !!! |%% 
%%%%%%%%%%%%%%%%%%%%%%%%%%%%%%%%%%%%%%%%%%%%%%%%%%%%%%%%%%%%%%%%%%%%%%%%%%%%%%%%%%%%%%%%%%%%%%%%%%%%%%%%%%%%%%%%%%%%%%%%%%%%%%%%%%%%%%%
%%| SET DOCUMENT PARAMTER HERE |%%
\documentclass[11pt]{article}

%%| INCLUDE PACKAGES HERE |%%
\usepackage[utf8]{inputenc}
\usepackage[paper=a4paper,left=25mm,right=25mm,top=25mm,bottom=25mm]{geometry}
\usepackage[english, ngerman]{babel}
\usepackage{graphicx}
\usepackage{amsmath}
\usepackage{amssymb}
\usepackage[autostyle=true,german=quotes]{csquotes}
\usepackage[backend=biber, style=ieee]{biblatex}
\usepackage{hyperref}
\usepackage{setspace}
\usepackage{microtype}

%%| SET PACKAGE PARAMETERS HERE |%%
\hypersetup{                                                                        
	colorlinks,
	citecolor   = black,
	filecolor   = black,
	linkcolor   = black,
	urlcolor    = black,
	pdftitle    = {Seminararbeit},                                               % SET FILE PARAMETERS HERE (SHOWN IN PDF PROPERTIES)
	pdfsubject  = {Dynamische Programmanalysen für nebenläufige Programme - Data Race Prediction mit TSan V2},                                       % ...
	pdfauthor   = {Frank Ling},                                                 % ...
	pdfkeywords = {Seminararbeit, Seminar} ,                                    % ...
	pdfcreator  = {pdflatex},           
	pdfproducer = {LaTeX with hyperref}
}
\setstretch{1.25}                                                                   % SET GENERAL LINE SPACING HERE

%%| SET OTHER PARAMETERS HERE |%%
\addbibresource{Document/Bibliography/ref.bib}

%%%%%%%%%%%%%%%%%%%%%%%%%%%%%%%%%%%%%%%%%%%%%%%%%%%%%%%%%%%%%%%%%%%%%%%%%%%%%%%%%%%%%%%%%%%%%%%%%%%%%%%%%%%%%%%%%%%%%%%%%%%%%%%%%%%%%%%
%%| TITLE PAGE |%%
\begin{document}
	\begin{titlepage}
		\begin{center}
			\includegraphics[width=0.55\textwidth]{images/hka-logo.png}\\[16ex]
			\huge{\textbf{Dynamische Programmanalysen für nebenläufige Programme - Data Race Prediction mit TSan V2}}\\[8ex]
			\LARGE{{Seminararbeit}}\\[14ex]
			\normalsize{}
			\begin{tabular}{lll}
				Student:            & \quad Frank Ling                                  & \\[2ex]
				Matrikelnummer:     & \quad 79496 & \\[2ex]     % NUMBER IS MAT.NR
				Universität:        & \quad Hochschule Karlsruhe – Technik und Wirtschaft   &       \\[2ex]
				Studiengang:        & \quad Informatik, Master                &       \\[2ex]
				Semester:           & \quad Sommersemester 2023                             &       \\[2ex]
				Dozent:             & \quad Prof. Martin Sulzmann                       &       \\[2ex]
				Bearbeitet am:      & \quad \today                                  &       \\[2ex]
			\end{tabular}
		\end{center}
	\end{titlepage}
	\newpage
	%%%%%%%%%%%%%%%%%%%%%%%%%%%%%%%%%%%%%%%%%%%%%%%%%%%%%%%%%%%%%%%%%%%%%%%%%%%%%%%%%%%%%%%%%%%%%%%%%%%%%%%%%%%%%%%%%%%%%%%%%%%%%%%%%%%%%%%
	%%| TABLE OF CONTENTS |%%
	\pagenumbering{gobble}    
	\tableofcontents
	\newpage
	\pagenumbering{arabic}
	%%%%%%%%%%%%%%%%%%%%%%%%%%%%%%%%%%%%%%%%%%%%%%%%%%%%%%%%%%%%%%%%%%%%%%%%%%%%%%%%%%%%%%%%%%%%%%%%%%%%%%%%%%%%%%%%%%%%%%%%%%%%%%%%%%%%%%%
	%%| WRITE TEXT HERE |%%
	\section{Einleitung}
	\begin{description}
		\item[data race] concurrent programs prone to data races, due to highly nondeterministic nature. 2 conflicting events next to each other in trace
		\item[conflicting event] 2 read/write events, at least one event is write event
		\item[dynamic data race prediction] predict trace orderings that exhibit data races
		\item[exhaustive predictive methods] identify as many orderings as possible
		\item[efficient predictive methods] \(O(n)\) runtime, compromise completeness and soundness
		\item[HB relation] events can be ordered by happens-before relation and if they can't that means they can be ordered in a way that they are next to each other in the trace $\rightarrow$ data race
		\item[vector clocks] used to represent happens-before relation, if incomparable then data race
		\item[epochs] vector clocks need \(O(n)\) time and space, instead epochs can be used which consist of time stamp j and thread id k $\rightarrow$ j\#k
	\end{description}
	
	\section{FastTrack Algorithmus TSan}
	\begin{itemize}
		\item FastTrack uses an optimized semi-adaptive version of epochs
	\end{itemize} \cite{cormac} \cite{sulzmann}
	
	\section{TSan Tool Beispiele Anwendung, Code}
	
	\section{Fazit}
	
	%%%%%%%%%%%%%%%%%%%%%%%%%%%%%%%%%%%%%%%%%%%%%%%%%%%%%%%%%%%%%%%%%%%%%%%%%%%%%%%%%%%%%%%%%%%%%%%%%%%%%%%%%%%%%%%%%%%%%%%%%%%%%%%%%%%%%%%
	%%| BIBLIOGRAPHY |%%                 
	\newpage                                          
	\printbibliography[heading= bibintoc, title={Literaturverzeichnis}]
	\newpage
	\listoffigures
	\addcontentsline{toc}{section}{Abbildungsverzeichnis}
	\newpage
	\listoftables
	\addcontentsline{toc}{section}{Tabellenverzeichnis}
\end{document}